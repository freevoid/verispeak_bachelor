\documentclass[utf8,usehyperref,14pt]{G7-32}
%[a4paper,titlepage,oneside,12pt]{gost/G7-32}

%\usepackage[pagebackref, bookmarks=true, bookmarksopen=false, pdfpagemode=UseOutlines, colorlinks=true, linkcolor=black, anchorcolor=black, citecolor=black, filecolor=black, menucolor=black, pagecolor=black, urlcolor=UrlColor]{hyperref}

\usepackage{float}

\TableInChaper % таблицы будут нумероваться в пределах раздела
\PicInChaper   % рисунки будут нумероваться в пределах раздела
\EqInChaper
\setlength\GostItemGap{2mm}% для красоты можно менять от 0мм

\newcommand{\Intro}{
    \Introduction
}

\newcommand{\Outtro}{
    \Conclusion
}

% Библиография по ГОСТу
\bibliographystyle{utf8gost71s}


% Опционно, требует  apt-get install scalable-cyrfonts.*
% и удаления одной строчки в cyrtimes.sty
% Сточку не удалять!
%\usepackage{cyrtimes}

%\usepackage[usenames]{color}
% Картинки и tikz
\usepackage[pdftex]{graphicx}
\usepackage{tikz}
\usepackage{pgfplots}
\pgfplotsset{compat=newest}
\usetikzlibrary{arrows,shapes,calc,decorations,chains,scopes,positioning,snakes,shadows}
\usepackage{tikz-er2}

% Некоторая русификация.
%\usepackage{misccorr}

\renewcommand{\labelitemi}{\normalfont\bfseries{--}}
\sloppy

% Оглавление в PDF
%\usepackage[
%bookmarks=true,
%colorlinks=true, linkcolor=black, anchorcolor=black, citecolor=black, menucolor=black,filecolor=black, urlcolor=black,
%unicode=true
%]{hyperref}

\newcommand{\Code}[1]{\texttt{#1}}
\newcommand{\Link}[1]{\mbox{\emph{#1}}}
\newcommand{\Soft}[1]{\mbox{\textbf{#1}}}
\usepackage{moreverb}

% Свои дополнения
%
% API Documentation for API Documentation
% Include File
%
% Generated by epydoc 3.0.1
% [Mon Dec  7 01:35:23 2009]
%
\usepackage{alltt, boxedminipage}
\setlength{\fboxrule}{2\fboxrule}
\newlength{\BCL} % base class length, for base trees.
\definecolor{py@keywordcolour}{rgb}{1,0.45882,0}
\definecolor{py@stringcolour}{rgb}{0,0.666666,0}
\definecolor{py@commentcolour}{rgb}{1,0,0}
\definecolor{py@ps1colour}{rgb}{0.60784,0,0}
\definecolor{py@ps2colour}{rgb}{0.60784,0,1}
\definecolor{py@inputcolour}{rgb}{0,0,0}
\definecolor{py@outputcolour}{rgb}{0,0,1}
\definecolor{py@exceptcolour}{rgb}{1,0,0}
\definecolor{py@defnamecolour}{rgb}{1,0.5,0.5}
\definecolor{py@builtincolour}{rgb}{0.58039,0,0.58039}
\definecolor{py@identifiercolour}{rgb}{0,0,0}
\definecolor{py@linenumcolour}{rgb}{0.4,0.4,0.4}
\definecolor{py@inputcolour}{rgb}{0,0,0}
% Prompt
\newcommand{\pysrcprompt}[1]{\textcolor{py@ps1colour}{\small\textbf{#1}}}
\newcommand{\pysrcmore}[1]{\textcolor{py@ps2colour}{\small\textbf{#1}}}
% Source code
\newcommand{\pysrckeyword}[1]{\textcolor{py@keywordcolour}{\small\textbf{#1}}}
\newcommand{\pysrcbuiltin}[1]{\textcolor{py@builtincolour}{\small\textbf{#1}}}
\newcommand{\pysrcstring}[1]{\textcolor{py@stringcolour}{\small\textbf{#1}}}
\newcommand{\pysrcdefname}[1]{\textcolor{py@defnamecolour}{\small\textbf{#1}}}
\newcommand{\pysrcother}[1]{\small\textbf{#1}}
% Comments
\newcommand{\pysrccomment}[1]{\textcolor{py@commentcolour}{\small\textbf{#1}}}
% Output
\newcommand{\pysrcoutput}[1]{\textcolor{py@outputcolour}{\small\textbf{#1}}}
% Exceptions
\newcommand{\pysrcexcept}[1]{\textcolor{py@exceptcolour}{\small\textbf{#1}}}
\newlength{\funcindent}
\newlength{\funcwidth}
\setlength{\funcindent}{1cm}
\setlength{\funcwidth}{\textwidth}
\addtolength{\funcwidth}{-2\funcindent}
\newlength{\varindent}
\newlength{\varnamewidth}
\newlength{\vardescrwidth}
\newlength{\varwidth}
\setlength{\varindent}{1cm}
\setlength{\varnamewidth}{.3\textwidth}
\setlength{\varwidth}{\textwidth}
\addtolength{\varwidth}{-4\tabcolsep}
\addtolength{\varwidth}{-3\arrayrulewidth}
\addtolength{\varwidth}{-2\varindent}
\setlength{\vardescrwidth}{\varwidth}
\addtolength{\vardescrwidth}{-\varnamewidth}
\newenvironment{Ventry}[1]%
 {\begin{list}{}{%
   \renewcommand{\makelabel}[1]{\texttt{##1:}\hfil}%
   \settowidth{\labelwidth}{\texttt{#1:}}%
   \setlength{\leftmargin}{\labelsep}%
   \addtolength{\leftmargin}{\labelwidth}}}%
 {\end{list}}
\definecolor{UrlColor}{rgb}{0,0.08,0.45}

\def\ssp{\def\baselinestretch{1.0}\normalsize}

\newcommand{\tt}{\ttfamily}
\newcommand{\it}{\itshape}

% QUOTATION
%   Fills lines
%   Indents paragraph
%
\def\quotation{\par\list{}{\ssp\listparindent 1.5em
    \itemindent\listparindent
    \rightmargin\leftmargin\parsep \z@ plus\p@}\item[]}
\let\endquotation=\endlist

% QUOTE -- same as quotation except no paragraph indentation,
%
\def\quote{\par\list{}{\ssp\rightmargin\leftmargin}\item[]}
\let\endquote=\endlist

\usepackage{indentfirst}
\usepackage{dot2texi}
\usepackage{gnuplottex}
\usepackage{pdflscape}
\usepackage{rotating}
\usepackage{amssymb}
\usepackage{amsmath}
\usepackage{pdfpages}
\usepackage{multirow}
%\usepackage{fix-cm}
%\usepackage{msc}
\usepackage[underline=true,rounded corners=false]{pgf-umlsd}

\newcommand{\executeiffilenewer}[3]{%
  \ifnum\pdfstrcmp{\pdffilemoddate{#1}}%
  {\pdffilemoddate{#2}}>0%
  {\immediate\write18{#3}}\fi%
}
\newcommand{\includesvg}[1]{%
  \executeiffilenewer{svg/#1.svg}{#1.pdf}%
  {inkscape -z -D --file=svg/#1.svg %
  --export-pdf=#1.pdf --export-latex}%
  \input{#1.tex}%
}

%\oddsidemargin 0.5cm
%\textwidth 16.5cm

\newcommand{\important}[1]{\emph{#1}}
\newcommand{\linux}{\mbox{GNU/Linux}}
\newcommand{\oslinux}{\mbox{ОС \linux}}

\providecommand{\refeq}[1]{(\ref{#1})}

\newenvironment{lscommand}%
    {\nopagebreak\par\small\addvspace{3.2ex plus 0.8ex minus 0.2ex}%
     \vskip -\parskip
     \noindent%
     \begin{tabular}{|l|}\hline\rule{0pt}{1em}\ignorespaces}%
    {\\\hline\end{tabular}\par\nopagebreak\addvspace{3.2ex plus 0.8ex
        minus 0.2ex}%
     \vskip -\parskip}

\newcommand{\inputdefspread}[1]{
    \linespread{1}
    \input{#1}
}

