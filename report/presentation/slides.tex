\documentclass[12pt]{article}

\usepackage[utf8]{inputenc}
\usepackage[english, russian]{babel}

% "А вот теперь --- слайды!"
\usepackage{slides}

% ШРИФТЫ
% Нужны рубленные шрифты -- раскомментируйте стоку ниже. 
% Нужны шрифты с засечками --- закомментируйте эту строку. 
% \renewcommand{\familydefault}{\sfdefault} % Переключает на рубленный шрифт.
% Шрифты Times и Arial, если стоит пакет cyrtimes. 
% Если он не стоит, результат будет плохой!
\usepackage{cyrtimespatched}
% Если нет cyrtimes, то попробуйте включить полужирный шрифт:
% \renewcommand{\seriesdefault}{b} % для шрифта с засечками, это предпочтительно
% \renewcommand{\seriesdefault}{sbc} % для рубленного шрифта


% Прочие пакетики
% Графика.
% Картинки и tikz
\usepackage[pdftex]{graphicx}
\usepackage{tikz}
\usepackage{pgfplots}
\pgfplotsset{compat=newest}
\usetikzlibrary{arrows,shapes,calc,decorations,chains,scopes,positioning,snakes,shadows}
\usepackage{tikz-er2}

\usepackage{amssymb}
\usepackage{amsmath}


\usepackage[underline=true,rounded corners=false]{pgf-umlsd}


\newcommand{\important}[1]{\emph{#1}}

% Настройка презентации
% Студент и руководитель.
\def\Student{Захаров Николай Игоревич}
\def\Advisor{Майков Константин Анатольевич}
% \def\Person
% \def\Affilation
% Титульный лист.
\def\Title{Программно-алгоритмический комплекс голосовой аутентификации пользователя}
\def\SubTitle{Квалификационная работа}
\input{title.tex}

% Верхний заголовок: пустой
% Нижний заголовок по-умолчанию:
% \lfoot{\Title} % слева
% \cfoot{} % цент пуст
% \rfoot{\thepage} % справа

% \renewcommand{\baselinestretch}{1.5}
% \linespread{1.6}

\begin{document}

\TitleSlide

% Команды section и section начинают новый слайд.

\section{Цель и задачи работы}

Создание программно-алгоритмического комплекса голосовой аутентификации пользователя.

\begin{enumerate}
\item Проанализировать существующие решения в области голосовой аутентификации;
\item Разработать программно-аппаратный комплекс, позволяющий принимать, обрабатывать голосовые данные, а также принимать решение об аутентификации;
\item Разработать схему хранения голосовых данных на сервере;
\item Обеспечить работу системы в условиях мультиплатформенной программно-аппаратной среды;
\item Исследовать качественные характеристики системы при наличии акустических помех.
\end{enumerate}

\section{Задача голосовой верификации}

Проверка статистической гипотезы для источника речи $S$:

\begin{equation}
\label{eq:hypothesis}
\left\{ 
    \begin{array}{lll}
        H_0 & : & \textrm{высказывание } X \textrm{ произнес } S;\\
        H_1 & : & \textrm{высказывание } X \textrm{ произнес \important{не} } S.\\
    \end{array}
\right.
\end{equation}

\section{Основные этапы работы системы голосовой аутентификации}

\begin{enumerate}
\item Нормализация входного речевого сигнала;
\item Выделение характерных признаков;
\item Построение модели источника речи (стадия обучения);
\item Принятие по построенной модели и новой входной последовательности одной из двух гипотез (\ref{eq:hypothesis}).
\end{enumerate}

\section{Диаграмма функциональных блоков процесса верификации}

\begin{figure}[h!]
\center{\includegraphics[width=\textwidth]{../include/idef0_main_dia}}
\end{figure}

\section{Нормализация речевого сигнала}

\begin{itemize}
    \item Удаление тишины;
    \item Частотная развертка (учет ограниченности спектра человеческой речи);
    \item Нормализация по средним (\important{mean normalization});
\end{itemize}

\section{Выделение характерных признаков}

Преобразование сегментированного сигнала в массив векторов характерных признаков:
\begin{itemize}
    \item Коэффициенты кода линейного предсказания (\important{LPCС});
    \item Коэффициенты Вейвлет-преобразования;
    \item Мел-частотные коэффициенты кепстра (\important{MFCC}).
\end{itemize}

\section{Диаграмма функциональных блоков обработки речи}

\begin{figure}[h!]
\center{\includegraphics[width=\textwidth]{../include/idef0_pre_dia}}
\end{figure}

\section{Подходы к моделированию источника речи}

\begin{itemize}
\item Алгоритмы сопоставления с образцом: динамического временного преобразования (\emph{DTW}), Евклидово расстояние, расстояние Махаланобиса.
\item Скрытые Марковские модели (\emph{HMM}, \emph{Hidden Markov Models});
\item Искусственные нейронные сети;
\item Смеси Гауссиан (\emph{GMM}, \emph{Gaussian Mixture Models}).
\end{itemize}

\section{Модели на основе смеси Гауссиан}

Моделируем плотность распределения вектора характерных признаков через взвешенную сумму $K$ нормальных распределений:

\begin{equation}
p(\vec x | \lambda) = \sum_{i=1}^K{\omega_i p_i(\vec x)},
\end{equation}

\noindent где $K$ -- количество компонент,\\
$\vec x$ -- вектор характерных признаков,\\
$\omega_i$ -- вес $i$-ой компоненты, \\
$p_i$ -- плотность распределения $i$-ой компоненты.

\section{Пример моделирования распределения с помощью смеси Гауссиан}

\begin{figure}[h!]
\center{\includegraphics[height=.75\textheight]{../include/gmm_merged_full_svg}}
\end{figure}

\section{Проблемы при использовании смеси Гауссиан}

\begin{itemize}
\item Выбор количества компонент смеси;
\item Инициализация параметров модели перед обучением;
\item Исчезновение порядка в матрице ковариаций;
\item Вычисление правдоподобия альтернативной гипотезы.
\end{itemize}

\section{Проблема вычисления $p(X|H_1)$}

\begin{itemize}
\item Универсальная фоновая модель: обучение GMM по нескольким часам речевых данных, сбалансированных по гендерному типу, возрасту, типу окружения;
\item Модели когортов: нахождение моделей, близких по мере правдоподобия $p(X|H_0)$ к модели целевого источника речи.
\end{itemize}

\section{Принятие решения}

Предположим, что меры правдоподобия обеих гипотез (\ref{eq:hypothesis}) известны. Рассмотрим \important{отношение правдоподобия}:

\begin{equation}
\label{eq:lr}
LR_{H_0, H_1} = \frac{p(X|H_0)}{p(X|H_1)},
\end{equation}

\begin{equation}
\label{eq:decision}
\textrm{Решение } = \left\{ 
    \begin{array}{ll}
        H_0, & LR_{H_0, H_1} > \Theta_{S} \\
        H_1, & LR_{H_0, H_1} \leq \Theta_{S}, \\
    \end{array}
\right.
\end{equation}

\noindent где $\Theta_{S}$ -- величина порога верификации.

\section{Диаграмма вариантов использования}

\begin{figure}[h!]
\center{\includegraphics[height=.8\textheight]{../include/use_cases_dia}}
\end{figure}

\section{Диаграмма сущность-связь системы хранения}

\begin{figure}[h!]
%\center{\includegraphics[height=.8\textheight]{../include/er_main_tikz}}
\end{figure}

\section{Диаграмма состояний для процесса регистрации в системе}

\begin{figure}[h!]
\center{\includegraphics[height=.8\textheight]{../include/enrollment_server_sd_dia}}
\end{figure}

\section{Диаграмма последовательности для процесса регистрации в системе}
\begin{figure}[hp!]
    \center{
        \fontsize{12}{14}\selectfont
        \begin{sequencediagram}
    \newthread{usr}{:Интерфейс}
    \newinst[2cm]{serverapi}{:Контроллер}
    \newinst[2.1cm]{session}{:Сессия}
    \newinst{fs}{:ФС}
    \newthread[1.5cm]{core}{:Ядро системы}

    % obtain session id
    \begin{call}{usr}{register()}{serverapi}{session\_id}
        \begin{call}{serverapi}{set\_state(created)}{session}{}
        \end{call}
    \end{call}
 
    % upload utterance
    \begin{sdloop}{Запись данных для обучения}
    \begin{call}{usr}{*[10..15]upload(data)}{serverapi}{success}
        \begin{call}{serverapi}{save\_upload\_wav(filename, data)}{fs}{}
        \end{call}
        \begin{call}{serverapi}{add\_wav(fname)}{session}{}
        \end{call}
    \end{call}
    \end{sdloop}

    % confirm enrollment
    \begin{call}{usr}{confirm()}{serverapi}{started}
        \begin{call}{serverapi}{set\_state(started)}{session}{}
        \end{call}
        \begin{call}{serverapi}{enroll(session\_id)}{core}{}
        \end{call}
    \end{call}

    \prelevel
    \begin{call}{core}{get\_session\_context(id)}{session}{context}
    \end{call}

    \begin{call}{core}{read\_wav(fnames)}{fs}{wavfiles}
    \end{call}

    \begin{callself}[2]{core}{enroll()}{}
    \end{callself}

    \begin{call}{core}{set\_state(finished)}{session}{}
    \end{call}

    \prelevel\prelevel
    % monitor
    \begin{sdloop}{Мониторинг процесса}
        \begin{call}{usr}{*[state != finished]get\_state()}{serverapi}{state}
            \begin{call}{serverapi}{get\_state()}{session}{state}
            \end{call}
        \end{call}
    \end{sdloop}

\end{sequencediagram}

    }
    \caption{Диаграмма последовательности: процесс регистрации в системе}
    \label{fig:seq_enrollment}
\end{figure}

\section{Исследование показателей качества системы}

В качестве характеристик качества системы верификации принимаются вероятности совершения ошибок I-ого и II-ого рода:
\begin{enumerate}
\item Аутентификация завершилась неудачно для аутентичного источника речи (ложный отказ);
\item Аутентификация прошла успешно, но источник речи в действительности не аутентичен зарегистрированному (несанкционированный вход).
\end{enumerate}

Требуемый баланс между этими показателями достигается путем варьирования значения порога аутентификации в~(\ref{eq:decision}).
Частота ошибок (\important{Equal Error Rate}) -- значение вероятности при таком $\Theta$, когда вероятности ошибок I-ого и II-ого рода равны.

\end{document}

