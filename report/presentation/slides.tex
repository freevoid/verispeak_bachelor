\documentclass[utf8,usehyperref,14pt]{G7-32}
%[a4paper,titlepage,oneside,12pt]{gost/G7-32}

%\usepackage[pagebackref, bookmarks=true, bookmarksopen=false, pdfpagemode=UseOutlines, colorlinks=true, linkcolor=black, anchorcolor=black, citecolor=black, filecolor=black, menucolor=black, pagecolor=black, urlcolor=UrlColor]{hyperref}

\usepackage{float}

\TableInChaper % таблицы будут нумероваться в пределах раздела
\PicInChaper   % рисунки будут нумероваться в пределах раздела
\EqInChaper
\setlength\GostItemGap{2mm}% для красоты можно менять от 0мм

\newcommand{\Intro}{
    \Introduction
}

\newcommand{\Outtro}{
    \Conclusion
}

% Библиография по ГОСТу
\bibliographystyle{utf8gost71s}


% Опционно, требует  apt-get install scalable-cyrfonts.*
% и удаления одной строчки в cyrtimes.sty
% Сточку не удалять!
%\usepackage{cyrtimes}

%\usepackage[usenames]{color}
% Картинки и tikz
\usepackage[pdftex]{graphicx}
\usepackage{tikz}
\usepackage{pgfplots}
\pgfplotsset{compat=newest}
\usetikzlibrary{arrows,shapes,calc,decorations,chains,scopes,positioning,snakes,shadows}
\usepackage{tikz-er2}

% Некоторая русификация.
%\usepackage{misccorr}

\renewcommand{\labelitemi}{\normalfont\bfseries{--}}
\sloppy

% Оглавление в PDF
%\usepackage[
%bookmarks=true,
%colorlinks=true, linkcolor=black, anchorcolor=black, citecolor=black, menucolor=black,filecolor=black, urlcolor=black,
%unicode=true
%]{hyperref}

\newcommand{\Code}[1]{\texttt{#1}}
\newcommand{\Link}[1]{\mbox{\emph{#1}}}
\newcommand{\Soft}[1]{\mbox{\textbf{#1}}}
\usepackage{moreverb}

% Свои дополнения
\input{api.tex}
\usepackage{indentfirst}
\usepackage{dot2texi}
\usepackage{gnuplottex}
\usepackage{pdflscape}
\usepackage{rotating}
\usepackage{amssymb}
\usepackage{amsmath}
\usepackage{pdfpages}
\usepackage{multirow}
%\usepackage{fix-cm}
%\usepackage{msc}
\usepackage[underline=true,rounded corners=false]{pgf-umlsd}

\newcommand{\executeiffilenewer}[3]{%
  \ifnum\pdfstrcmp{\pdffilemoddate{#1}}%
  {\pdffilemoddate{#2}}>0%
  {\immediate\write18{#3}}\fi%
}
\newcommand{\includesvg}[1]{%
  \executeiffilenewer{svg/#1.svg}{#1.pdf}%
  {inkscape -z -D --file=svg/#1.svg %
  --export-pdf=#1.pdf --export-latex}%
  \input{#1.tex}%
}

%\oddsidemargin 0.5cm
%\textwidth 16.5cm

\newcommand{\important}[1]{\emph{#1}}
\newcommand{\linux}{\mbox{GNU/Linux}}
\newcommand{\oslinux}{\mbox{ОС \linux}}

\providecommand{\refeq}[1]{(\ref{#1})}

\newenvironment{lscommand}%
    {\nopagebreak\par\small\addvspace{3.2ex plus 0.8ex minus 0.2ex}%
     \vskip -\parskip
     \noindent%
     \begin{tabular}{|l|}\hline\rule{0pt}{1em}\ignorespaces}%
    {\\\hline\end{tabular}\par\nopagebreak\addvspace{3.2ex plus 0.8ex
        minus 0.2ex}%
     \vskip -\parskip}

\newcommand{\inputdefspread}[1]{
    \linespread{1}
    \input{#1}
}


\usepackage{beamerthemesplit}

\mode<presentation>
{
  \usetheme{Ilmenau}
  \setbeamercovered{transparent}
  % or whatever (possibly just delete it)
}
\useoutertheme{smoothbars}

\title{Система управления ресурсами}

\author{Захаров Н.И., Ковега Д.Н.}
\institute[МГТУ им.~Н.~Э.~Баумана]{%
    Кафедра <<Программное обеспечение ЭВМ и информационные технологии>>\\
    Московский Государственный Технический Университет имени Н.~Э.~Баумана}

\date{\today}

\begin{document}

\frame{\titlepage}

\section*{Содержание}
\frame{\tableofcontents}

\section[Цель]{Цель и задачи проекта}
\frame {
\frametitle{Цель проекта}
Целью данного проекта является создание программного комплекса, позволяющего управлять доступом к различным системам контроля версий и системам управления проектами для организации хранилища исходного кода студенческих проектов и создания условий для разработки (в том числе совместной).
}

\frame{
\frametitle{Задачи}
    Для достижения данной цели необходимо решить следующие задачи:
    \begin{itemize}
    \item Исследовать существующие системы управления проектами, способы их конфигурирования;
    \item Разработать и реализовать модули, предоставляющие абстрактный интерфейс для конфигурирования выбранных систем;
    \item Реализовать комплекс, включающий в себя базу данных студентов, курсов, заданий, для создания и управления учебными проектами;
    \item Разработать систему нотификации, позволяющую оповещать студентов о включении их в проекты и предоставлении им ресурсов.
    \end{itemize}

}

\section{Обоснование задачи}
\begin{frame}{Актуальность}{Актуальность выбранной темы}
    Выбранная задача является актуальной по следующим причинам:
    \begin{itemize}
    \item Отсутствие единого электронного хранилища исходных кодов студенческих проектов;
    \item Отсутствие единого интерфейса для удаленного взаимодействия руководителя (преподавателя) и студента при выполнении проектов;
    \item Отсутствие базы для изучения основ командной разработки и управления проектом;
    \end{itemize}
\end{frame}

\section{Структура комплекса}

\begin{frame}{Общая архитектура}{Общая архитектура комплекса}

    \begin{figure}
        \center{\includegraphics[width=0.8\textwidth]{../include/main_arch_dia.pdf}}
        \label{fig:main_arch}
    \end{figure}
\end{frame}

\begin{frame}{ER-диаграмма основных сущностей}{ER-диаграмма основных сущностей подсистемы, представляющей учебный процесс}

    \begin{figure}
        \center{\includegraphics[height=0.8\textheight]{../include/er_general_tikz.pdf}}
        \label{fig:er:general}
    \end{figure}

\end{frame}

\begin{frame}{ER-диаграмма основных сущностей}{Диаграмма сущность-связь для поддерживаемых СКВ и СУП}

\begin{figure}
    \center{\includegraphics[width=0.8\textwidth]{../include/er_subtables_tikz.pdf}}
    \label{fig:er:subtables}
\end{figure}

\end{frame}

\section{Основные алгоритмы}

\begin{frame}{Создание проекта}{Блок-схема алгоритма создания нового проекта}
\begin{figure}[h!]
    \center{\includegraphics[height=0.8\textheight]{../include/project_create_fc_dia.pdf}}
    \label{fig:fc:project_create}
\end{figure}
\end{frame}

\begin{frame}{Установка активного статуса}{Блок-схема алгоритма установки активного статуса проекта}
\begin{figure}[h!]
    \center{\includegraphics[height=0.8\textheight]{../include/project_set_active_fc_dia.pdf}}
    \label{fig:fc:project_set_active}
\end{figure}
\end{frame}

\section*{Заключение}

\begin{frame}{Заключение}{В результате выполнения проекта были выполнены следующие задачи}
    \begin{itemize}
    \item Реализован программный комплекс, предназначенный для управления проектами, выполняющихся в рамках учебного процесса;
    \item Реализованы модули, предоставляющие абстрактный интерфейс для конфигурации СУП \Soft{Trac} и СКВ \Soft{Git}, \Soft{Subversion};
    \item Разработана база данных, полностью описывающая выполнение заданий в рамках учебного процесса;
    \item Разработана система оповещения студентов о включении их в проекты и предоставлении им ресурсов;
    \item Реализована гибкая система прав как на доступ к СУП и СКВ, так и на администрирование системой. 
    \end{itemize}
\end{frame}

\end{document}

