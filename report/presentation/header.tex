\documentclass{beamer}
%[a4paper,titlepage,oneside,12pt]{gost/G7-32}

%\usepackage[pagebackref, bookmarks=true, bookmarksopen=false, pdfpagemode=UseOutlines, colorlinks=true, linkcolor=black, anchorcolor=black, citecolor=black, filecolor=black, menucolor=black, pagecolor=black, urlcolor=UrlColor]{hyperref}
\usepackage[T2A]{fontenc}
\usepackage[utf8x]{inputenc}
\usepackage[english,russian]{babel}


\usepackage{float}

% Библиография по ГОСТу
\bibliographystyle{utf8gost71s}


% Опционно, требует  apt-get install scalable-cyrfonts.*
% и удаления одной строчки в cyrtimes.sty
% Сточку не удалять!
%\usepackage{cyrtimes}

%\usepackage[usenames]{color}
% Картинки и tikz

% Некоторая русификация.
%\usepackage{misccorr}

\sloppy

% Оглавление в PDF
%\usepackage[
%bookmarks=true,
%colorlinks=true, linkcolor=black, anchorcolor=black, citecolor=black, menucolor=black,filecolor=black, urlcolor=black,
%unicode=true
%]{hyperref}

\newcommand{\Code}[1]{\texttt{#1}}
\newcommand{\Link}[1]{\mbox{\emph{#1}}}
\newcommand{\Soft}[1]{\mbox{\textbf{#1}}}
\usepackage{moreverb}

% Свои дополнения
\usepackage{indentfirst}
\usepackage{dot2texi}
\usepackage{gnuplottex}
\usepackage{pdflscape}
\usepackage{rotating}
\usepackage{amssymb}
\usepackage{amsmath}
\usepackage{pdfpages}
\usepackage{multirow}

%\oddsidemargin 0.5cm
%\textwidth 16.5cm

\newcommand{\important}[1]{\emph{#1}}
\newcommand{\linux}{\mbox{GNU/Linux}}
\newcommand{\oslinux}{\mbox{ОС \linux}}

\providecommand{\refeq}[1]{(\ref{#1})}

\newenvironment{lscommand}%
    {\nopagebreak\par\small\addvspace{3.2ex plus 0.8ex minus 0.2ex}%
     \vskip -\parskip
     \noindent%
     \begin{tabular}{|l|}\hline\rule{0pt}{1em}\ignorespaces}%
    {\\\hline\end{tabular}\par\nopagebreak\addvspace{3.2ex plus 0.8ex
        minus 0.2ex}%
     \vskip -\parskip}

\newcommand{\inputdefspread}[1]{
    \linespread{1}
    \input{#1}
}

