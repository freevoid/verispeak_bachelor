\Abbreviations %% Список обозначений и сокращений в тексте
\begin{description}
\item[MFCC] Mel-frequency cepstral coefficients. В обработке сигналов, мел-частотные кепстральные коэффициенты представляют краткосрочный спектр мощности звука, основанный на линейном косинусном преобразовании спектра мощности мел-частот;
\item[GMM] Gaussian mixture model. Модель смеси гауссиан -- статистическая модель, позволяющая оценить плотность распределения случайной величины, представляя функцию плотности распределения в виде взвешенной суммы функций нормального распределения (компонент смеси);
\item[UBM] Universal background model -- универсальная фоновая модель. В голосовой аутентификации -- гипотетическая модель, отражающая общие характеристики речи для всех возможных источников речи (а также реальные аппроксимации данной модели);
\item[HMM] Hidden Markov Models -- скрытая Марковская модель. Статистическая модель, имитирующая работу процесса, похожего на марковский процесс с неизвестными параметрами, и задачей ставится оценка неизвестных параметров на основе наблюдаемых. Полученные параметры могут быть использованы в дальнейшем анализе, например, для распознавания образов;
\item[ANN] Artificial Neural Networks -- искуственные нейронные сети.
\end{description}

