\chapter{Конструкторский раздел}
Данный раздел содержит описание архитектуры разрабатываемого программного комплекса, а также ключевые алгоритмы, описывающие логику разрабатываемой системы.

\section{Общая архитектура разрабатываемого комплекса}
\label{sec:main_arch}

В разрабатываемом программном комплексе можно выделить три основные подсистемы:
\begin{itemize}
\item Подсистема голосовой аутентификации -- ядро программного комплекса, реализующее подходы, выбранные в результате обзора существующих решений в разделе~\ref{sec:overview};
\item Подсистема хранения речевых данных пользователей (база данных);
\item Интерфейс пользователя -- приложение, позволяющее пользователю производить регистрацию и аутентификацию в системе, реализующее передачу данных с микрофона пользователя на сервер (для сохранения в базе данных и последующей обработки).
\end{itemize}

Общая архитектура комплекса представлена на рисунке~\ref{fig:main_arch}.

\begin{figure}
    \center{\includegraphics[width=0.8\textwidth]{include/main_arch_dia.pdf}}
    \caption{Общая архитектура разрабатываемого программного комплекса}
    \label{fig:main_arch}
\end{figure}

\section{Варианты использования}

На рисунке~\ref{fig:use_cases} представлена диаграмма вариантов использования системы.

\begin{figure}[htp!]
    \center{\includegraphics[width=0.8\textwidth]{include/use_cases_dia.pdf}}
    \caption{Диаграмма вариантов использования системы}
    \label{fig:use_cases}
\end{figure}

\section{Подсистема голосовой аутентификации}

На рисунке~\ref{fig:fsm_learning} представлена диаграмма состояний для процесса регистрации пользователя в системе голосовой аутентификации (то есть процесса обучения новой модели по речевым данным пользователя).

\begin{figure}
    \center{\includegraphics[width=0.8\textwidth]{include/learning_fsm_times_dia.pdf}}
    \caption{Диаграмма состояний для процесса обучения модели}
    \label{fig:fsm_learning}
\end{figure}

На рисунке~\ref{fig:idef0_main} показана диаграмма функциональных блоков для процесса верификации.

\begin{figure}
    \center{
    \begin{sideways}
        \includegraphics[width=0.9\textheight]{include/idef0_main_dia.pdf}
    \end{sideways}
    }
    \caption{Диаграмма функциональных блоков A0: процесс верификации}
    \label{fig:idef0_main}
\end{figure}

\begin{figure}
    \center{
    \begin{sideways}
    \includegraphics[width=0.9\textheight]{include/idef0_pre_dia.pdf}
    \end{sideways}
    }
    \caption{Диаграмма функциональных блоков A1: подготовка входных данных}
    \label{fig:idef0_pre}
\end{figure}

\begin{figure}
    \center{
    \begin{sequencediagram}
    \newthread{usr}{:Интерфейс}
    \newinst[2cm]{serverapi}{:Контроллер}
    \newinst[2.1cm]{session}{:Сессия}
    \newinst{fs}{:ФС}
    \newthread[1.5cm]{core}{:Ядро системы}

    % obtain session id
    \begin{call}{usr}{register()}{serverapi}{session\_id}
        \begin{call}{serverapi}{set\_state(created)}{session}{}
        \end{call}
    \end{call}
 
    % upload utterance
    \begin{sdloop}{Запись данных для обучения}
    \begin{call}{usr}{*[5..10]upload(file)}{serverapi}{success}
        \begin{call}{serverapi}{save\_upload\_wav(filename, file)}{fs}{}
        \end{call}
        \begin{call}{serverapi}{add\_wav(fname)}{session}{}
        \end{call}
    \end{call}
    \end{sdloop}

    % confirm enrollment
    \begin{call}{usr}{confirm()}{serverapi}{started}
        \begin{call}{serverapi}{set\_state(started)}{session}{}
        \end{call}
        \begin{call}{serverapi}{enroll(session\_id)}{core}{}
        \end{call}
    \end{call}

    \prelevel
    \begin{call}{core}{get\_session\_context(id)}{session}{context}
    \end{call}

    \begin{call}{core}{read\_wav(fnames)}{fs}{wavfiles}
    \end{call}

    \begin{callself}[2]{core}{train()}{}
    \end{callself}

    \begin{call}{core}{set\_state(finished)}{session}{}
    \end{call}

    \prelevel\prelevel
    % monitor
    \begin{sdloop}{Мониторинг процесса}
        \begin{call}{usr}{*[state != finished]get\_state()}{serverapi}{state}
            \begin{call}{serverapi}{get\_state()}{session}{state}
            \end{call}
        \end{call}
    \end{sdloop}

\end{sequencediagram}

    }
    \caption{Диаграмма последовательности: процесс регистрации в системе}
    \label{fig:seq_enrollment}
\end{figure}


\begin{figure}
    \center{
    \begin{sequencediagram}
    \newthread{usr}{:Интерфейс}
    \newinst[2cm]{serverapi}{:Контроллер}
    \newinst[2.1cm]{session}{:Сессия}
    \newinst{fs}{:ФС}
    \newthread[1.5cm]{core}{:Ядро системы}

    % obtain session id
    \begin{call}{usr}{verify()}{serverapi}{session\_id}
        \begin{call}{serverapi}{set\_state(created)}{session}{}
        \end{call}
    \end{call}
 
    % upload utterance
    \begin{call}{usr}{upload(data)}{serverapi}{started}
        \begin{call}{serverapi}{save\_upload\_wav(filename, data)}{fs}{}
        \end{call}
        \begin{call}{serverapi}{add\_wav(fname)}{session}{}
        \end{call}
        \begin{call}{serverapi}{set\_state(started)}{session}{}
        \end{call}
        \begin{call}{serverapi}{verificate(session\_id)}{core}{}
        \end{call}
    \end{call}

    \prelevel
    \begin{call}{core}{get\_session\_context(id)}{session}{context}
    \end{call}

    \begin{call}{core}{read\_wav(fname)}{fs}{wavfile}
    \end{call}
    \begin{call}{core}{read\_model(fname)}{fs}{model}
    \end{call}

    \begin{callself}[2]{core}{verify()}{result}
    \end{callself}

    \begin{call}{core}{set\_state(finished, result)}{session}{}
    \end{call}

    \prelevel\prelevel\prelevel\prelevel\prelevel\prelevel\prelevel
    % monitor
    \begin{sdloop}{Мониторинг процесса}
        \begin{call}{usr}{*[state != finished]get\_state()}{serverapi}{state}
            \begin{call}{serverapi}{get\_state()}{session}{state}
            \end{call}
        \end{call}
    \end{sdloop}

\end{sequencediagram}

    }
    \caption{Диаграмма последовательности: процесс аутентификации}
    \label{fig:seq_verification}
\end{figure}

\section{Подсистема хранения голосовых данных}

На рисунке~\ref{fig:er_main} представлена диаграмма сущность-связь для разрабатываемой базы данных. Рассмотрим подробнее сущности, представленные на данной диаграмме:

\begin{itemize}
\item Пользователь;
\item Модель говорящего;
\item Сессия записи;
\item Сессия верификации;
\item Сессия регистрации;
\item Записанное высказывание;
\item Обучение;
\item Аутентификация.
\end{itemize}

\begin{figure}
    %
\tikzstyle{every entity} = [top color=white, bottom color=blue!30, 
                            draw=blue!50!black!100, drop shadow]
\tikzstyle{every weak entity} = [drop shadow={shadow xshift=.7ex, 
                                 shadow yshift=-.7ex}]
\tikzstyle{every attribute} = [top color=white, bottom color=yellow!20, 
                               draw=yellow, node distance=1cm, drop shadow]
\tikzstyle{every relationship} = [top color=white, bottom color=red!20, 
                                  draw=red!50!black!100, drop shadow,
                                  font=\footnotesize]
\tikzstyle{every isa} = [top color=white, bottom color=green!20, 
                         draw=green!50!black!100, drop shadow]




\begin{comment}\end{comment}%\tikzstyle{every relationship} = [font=\footnotesize]

\newcommand*{\drawER}[2]{
\scalebox{#1}{
\begin{tikzpicture}
    \node [entity]      (speaker) {Источник речи};
    \node [attribute]   (sid)       [above=of speaker]        {\key{ID}} edge (speaker);
    \node [attribute]   (sname)     [above right=of speaker]  {Имя/Отчество} edge (speaker);
    \node [attribute]   (sfname)    [right=of speaker, xshift=1ex]  {Фамилия} edge (speaker);
    \node [attribute]   (susername) [above left=of speaker, xshift=3ex] {Имя учетной записи} edge (speaker);
    \node [attribute]   (semail)    [left=of speaker]   {E-mail} edge (speaker);

    \node [relationship] (beenrecorded) [below left=of speaker, xshift=-10ex]  {Участвовал} edge (speaker);

    \node [entity]      (recordsession) [below left=of beenrecorded] {Сессия записи} edge (beenrecorded);
    \node [attribute]   (rsid)          [left=of recordsession]        {\key{ID Сессии}} edge (recordsession);
    \node [attribute]   (rscreated)     [above=of recordsession, xshift=-4ex]        {Дата создания} edge (recordsession);
    \node [attribute]   (rscreated)     [right=of recordsession]        {IP-адрес инициатора} edge (recordsession);

    \node [ident relationship] (sessionutterance) [below left=of recordsession, yshift=-10ex, xshift=-1ex] {Содержит}   edge (recordsession);

    \node [weak entity]      (uploadedutterance) [below=of sessionutterance] {Записанная речь} edge [total] (sessionutterance);
    \node [attribute]   (uufilename)    [below=of uploadedutterance, yshift=-3em]        {Путь к файлу} edge (uploadedutterance);
    \node [attribute]   (uuuploadtime)  [below right=of uploadedutterance, yshift=2em] {Дата создания} edge (uploadedutterance);
    \node [attribute]   (uuid)  [below right=of uploadedutterance, yshift=-2em] {\key{ID}} edge (uploadedutterance);

    \node [ident relationship] (havemodel)    [below right=of speaker, xshift=10ex]  {Имеет} edge (speaker);

    \node [weak entity]      (speakermodel)  [below right=of havemodel]     {Модель источника речи} edge [total] (havemodel);
    \node [attribute]   (smfilename)    [left=of speakermodel]        {Путь к файлу} edge (speakermodel);
    \node [attribute]   (smid)       [above=of speakermodel]        {\key{ID}} edge (speakermodel);

    \node [relationship]  (session-learning) [below right=of recordsession] {Задействована в} edge (recordsession);

    \node [entity]      (learning)      [right=of session-learning] {Процесс обучения} edge (session-learning);
    \node [attribute]   (lid)        [above=of learning, yshift=-3ex, xshift=-10ex]        {\key{ID}} edge (learning);
    \node [attribute]   (lstate)        [above=of learning, yshift=-3ex, xshift=7ex]        {Состояние} edge (learning);
    \node [attribute]   (lstart)        [below right=of learning, xshift=-10ex]        {Дата начала} edge (learning);
    \node [attribute]   (lfinish)       [below left=of learning, xshift=10ex]        {Дата конца} edge (learning);

    \node [relationship]  (model-learning) [right=of learning, xshift=-1ex, text width=15ex, text centered] {Получена в результате} edge (learning);
    \draw [link] (model-learning) -- (speakermodel);

    \node [ident relationship] (model-verificator) [below=of speakermodel] {Использована в} edge (speakermodel);

    \node [weak entity] (verificator) [below=of model-verificator] {Верификатор} edge [total] (model-verificator);
    \node [attribute] (vfid) [above left=of verificator, xshift=3ex, yshift=-4ex] {\key{ID}} edge (verificator);
    \node [attribute] (vftreshhold) [left=of verificator] {Порог вхождения} edge (verificator);

    \node [ident relationship] (verificator-ubm) [below=of verificator] {Использует} edge [total] (verificator);
    
    \node [entity] (ubm) [below=of verificator-ubm] {Универсальная фоновая модель} edge (verificator-ubm);
    \node [attribute] (ubmid) [above left=of ubm, xshift=5ex, yshift=-3ex] {\key{ID}} edge (ubm);
    \node [attribute] (ubmpath) [left=of ubm] {Путь к файлу} edge (ubm);

    \node [relationship] (verificator-verification) [below left=of verificator] {Использован в} edge (verificator);

    \node [weak entity]      (verification)  [below=of learning, yshift=-25ex, xshift=-5ex] {Процесс верификации} edge (verificator-verification);
    \node [attribute]   (vstate)        [above=of verification, yshift=-0ex]        {Состояние} edge (verification);
    \node [attribute]   (vresult)       [above right=of verification, xshift=-3ex]        {Результат} edge (verification);
    \node [attribute]   (vstart)        [below right=of verification, xshift=-10ex]        {Дата начала} edge (verification);
    \node [attribute]   (vfinish)       [below left=of verification, xshift=10ex]        {Дата конца} edge (verification);
    \node [attribute]   (vid)           [left=of verification]        {\key{ID}} edge (verification);

    \node [ident relationship] (session-verification)  [below=of recordsession] {Задействована в}   edge (recordsession);
    \draw [link] (session-verification) [total] -- (verification);

\end{tikzpicture}}}

\begin{comment}
    \draw [link]        (humansecret)   [total] -- ++(-10ex, 0) |- (secret);
    \draw [link] (coursetorole) -| (courserole);
\end{comment}


    %\center{\drawER{0.65}{1.0cm}}
    \center{\includegraphics[width=\textwidth]{include/er_main_tex.pdf}}
    \caption{Диаграмма сущность-связь подсистемы хранения голосовых данных}
    \label{fig:er_main}
\end{figure}

%Следует отметить, что база данных предназначена для хранения вспомогательной информации, тогда как непосредственно голосовые данные должны храниться 

\section{Основные алгоритмы}

