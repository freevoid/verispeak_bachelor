\chapter{Экспериментальный раздел}
\label{sec:experiment}

В данном разделе описаны проведенные эксперименты, призванные определить
точность разработанной системы при аутентификации пользователей, а также
определить необходимые для работы системы параметры.

\section{Планирование эксперимента}

\subsection{Определение точности системы аутентификации}

При анализе биометрических систем, точность системы
определяется вероятностями совершения ошибок I-ого и II-ого рода:

\begin{enumerate}
\item аутентификация завершилась неудачно для аутентичного источника речи (ложный отказ);
\item аутентификация прошла успешно, но источник речи в действительности не аутентичен зарегистрированному (несанкционированный вход).
\end{enumerate}

Требуемый баланс между этими показателями достигается путем варьирования
значения порога вхождения. Частота ошибок (\important{Equal Error Rate}) --
значение вероятности при таком пороге вхождения, когда вероятности ошибок I-ого
и II-ого рода равны.

Для более подробного отражения качественных характеристик системы используются
так называемые DET-кривые (Detection Error Trade-off curves)~\cite{Alvin97DET}. На оси $x$
DET-кривой отражается процент ложного отказа, при этом на оси $y$ отражается
процент ложного положительного решения (при одинаковом пороге). Варьируя
значение порога вхождения, получаем кривую. Данный метод позволяет качественно и
количественно сравнивать
показатели системы при изменении параметров.

\subsection{Описание экспериментов}

Для работы системы необходимо определить значения ряда параметров, в частности:

\begin{enumerate}

\item количество фраз для стадии обучения модели (см.
раздел~\ref{sec:construct:enrollment}). При создании персональной модели
источника речи, пользователю необходимо произнести ключевую фразу несколько раз,
чтобы накопить достаточное для обучения модели количество речевых данных.
Эксперимент позволит определить такое минимальное количество, при котором для
тестируемой выборки будут соблюдены требования, предъявляемые к точности
аутентификации;

\item величина порога вхождения (реализованный алгоритм голосовой верификации даёт
численную оценку правдоподобия проверяемой гипотезы, но для принятия решения
необходимо определить, достаточна ли полученная величина для того, чтобы с
уверенностью говорить о положительном результате аутентификации). От данной
величины напрямую зависит чувствительность системы, поэтому, если она выбирается
глобально, необходимо получить её экспериментально, то есть найти минимальное
значение, при котором исключаются ошибки второго рода (ложные срабатывания).

\item влияние применения алгоритма удаления тишины на точность аутентификации (алгоритм
рассмотрен в разделе~\ref{sec:construct:silence_remove}). Целью эксперимента
является определение параметров алгоритма, при которых достигается наибольшая
точность (в рамках имеющейся тестовой выборки).

\item количество компонент смеси Гауссиан. Применяемый метод для моделирования
источника речи параметризуется количеством компонент смеси. Чем больше это
количество, тем, теоретически, точнее ожидаемая точность аутентификации.
Эксперимент позволит определить минимальное количество, при котором
удовлетворяются требования точности.

\end{enumerate}

\subsection{Описание входных данных}

Для проведения экспериментов была составлена выборка, состоящая из 30 человек,
из них 16 мужчин и 15 женщин. Каждый человек записывал одну и ту же фразу, что
позволяет анализировать способность системы различать особенности речи отдельных
людей. Запись проходила в обстановке, максимально приближенной к условиям
эксплуатации программного комплекса: каждый источник речи использовал для записи
собственный микрофон и записывался в домашних условиях, удаленно. Средняя длина
записанной фразы: $2.913$ секунд. Количество записей в расчете на человека:
$60$.

При определении количества смеси Гауссиан, для каждого источника речи в тестовой выборке производилось обучение моделей из $k$ по 10 фразам (фразы для обучения). По данным предварительных испытаний, при $k>28$ и количестве тестовых данный менее 1 минуты обучение модели заканчивалось неудачей из-за ошибок округления, поэтому $k=\overline{1,28}$.
При определении необходимого количества фраз для обучения для каждого источника речи производилось обучение моделей по $n$ фразам, где $n=\overline{5, 20}$. Количество компонент выбиралось на основе результатов предыдущего эксперимента.
При изучении зависимости параметра алгоритма удаления тишины на точность аутентификации производилось обучение моделей из $k$ компонент по $n$ фразам, при этом входные данные обрабатывались алгоритмом удаления тишины с различными значениями параметра $\omega$: $\omega \in \{0, 0.05, \dots, 0.65, 0.7\}$.

Для каждой построенной модели строились показания ошибок I-ого и II-ого рода при значениях порога вхождения $\Theta \in \{-1500, -1450, -1400, \dots, 1450, 1500\}$. После вычисления данных показаний, результаты для всех источников речи и одинаковых факторов ($k$, $n$, $\omega$) усреднялись для получения средних показаний. По полученным показаниям находилось значение частоты равнозначной ошибки.

При вычислении вероятности ошибки I-ого рода использовались те записи источника речи, которые не были использованы при обучении.

При вычислении вероятности ошибки II-ого рода использовались записи всех источников речи в рамках выбранной субпопуляции.

\subsubsection*{Сбор данных}

Для сбора данных в программный комплекс был включен модуль, использующий
разработанную базу для записи и отправки голосовых данных от анонимных
пользователей и сохранения их в файловой системе с кратким описанием, что
позволило автоматизировать последующую обработку. Запись данных при этом
происходила удаленно. Форма для записи и отправки голосовых данных представлена
на рисунке~\ref{fig:ui:voice_upload}. Процесс записи аналогичен процессу
создания персональной голосовой модели, описанному в
разделе~\ref{sec:manual:enrollment}. После записи, участнику тестирования
необходимо было заполнить форму с описанием сессии, которая сохранялась в базе
данных для последующей обработки в процессе эксперимента.

\begin{figure}[ht!]
\center{\includegraphics[width=72mm]{static_include/voice_upload.png}}
\caption{Диалог записи и загрузки голосовых данных}
\label{fig:ui:voice_upload}
\end{figure}

\section{Анализ результатов}

На рисунке~\ref{fig:eer_from_k} отображена зависимость частоты равнозначной ошибки (значение вероятностей ошибок I-ого и II-ого рода, при таком пороге вхождения, при котором эти значения совпадают) от количества компонент в модели. Анализ данной зависимости позволяет заключить, что:
\begin{itemize}
\item аутентификация источников речи из мужской субпопуляции происходит с большей точностью;
\item при значении количества компонент, равному 20 наблюдается падение точности для обеих субпопуляций;
\item для мужской субпопуляции модель, состоящая из 25 Гауссиан позволяет получить наилучший результат по тестовой выборке;
\item для женской субпопуляции модель, состоящая из 21 Гауссиан позволяет получить наилучший результат по тестовой выборке.
\end{itemize}

\begin{figure}[ht!]
\center{\includegraphics[width=\textwidth]{static_include/eer_from_k.pdf}}
\caption{Зависимость частоты равнозначной ошибки от количества компонент в смеси Гауссиан}
\label{fig:eer_from_k}
\end{figure}

На рисунке~\ref{fig:eer_from_n} отображена зависимость частоты равнозначной ошибки от количества фраз, используемых при обучении модели. Анализ данной зависимости позволяет сделать следующий выводы:
\begin{itemize}
\item для мужской субпопуляции точность аутентификации достигает наибольшего значения при сессии регистрации, состоящей из 11 фраз;
\item для женской субпопуляции точность аутентификации достигает наибольшего значения при сессии регистрации, состоящей из 14 фраз;
\end{itemize}

\begin{figure}[ht!]
\center{\includegraphics[width=\textwidth]{static_include/eer_from_n.pdf}}
\caption{Зависимость частоты равнозначной ошибки от количества фраз, записанных при регистрации}
\label{fig:eer_from_n}
\end{figure}

На рисунке~\ref{fig:eer_from_w} отображена зависимость частоты равнозначной ошибки от значения параметра $\omega$ алгоритма удаления фрагментов записи, не содержащих речь. Параметр $\omega$ (см. раздел~\ref{sec:construct:silence_remove}) влияет на чувствительность алгоритма: при $\omega = 0$ алгоритм не изменяет сигнал, с увеличением $\omega$ увеличивается количество сегментов, которые алгоритм воспринимает как <<тишину>>. Анализ данной зависимости позволяет сделать вывод о том, что при значении $\omega > 0.3$ точность системы начинает понижаться, максимальная точность в рамках тестовой выборки достигается при $\omega = 0.25$ и равна $0.234\%$.

\begin{figure}[ht!]
\center{\includegraphics[width=\textwidth]{static_include/eer_from_w}}
\caption{Зависимость частоты равнозначной ошибки от значения параметра $\omega$}
\label{fig:eer_from_w}
\end{figure}


На рисунке~\ref{fig:overall} представлена зависимость вероятности ошибок I-ого и II-ого рода от значения порога вхождения при использовании вычисленных ранее параметров. Результаты обеих популяций были скомбинированы.
На пересечении кривых находится точка равнозначной ошибки (вероятность при этом равна 0.234\%). Как видно из графика, в точке равнозначной ошибки величина порога отрицательна. При этом, если установить допустимое значение вероятности ложных отказов в 20\% (вероятность отказа в аутентификации аутентичному источнику), то это позволит установить порог вхождения в 1300 единиц. В данном случае ошибка второго рода оказывается равной нулю (для тестируемой выборки).

\begin{figure}[ht!]
\center{\includegraphics[width=\textwidth]{static_include/overall}}
\caption{Зависимость ошибок аутентификации от значения порога вхождения}
\label{fig:overall}
\end{figure}

