\Outtro

В результате выполнения проекта были выполнены следующие задачи:

\begin{enumerate}
\item спроектирован и реализован программный комплекс, предназначенный для аутентификации
пользователя по голосу;
\item спроектирована и реализована подсистема сбора и хранения голосовых данных
пользователя;
\item спроектирован и реализован интерфейс администратора, позволяющий
контролировать и изменять хранимые в базе объекты, задействованные в процессе
аутентификации и регистрации;
\item разработан алгоритм удаления из аудиозаписи фрагментов, не содержащих
речи. В экспериментальной части исследовано влияние применения алгоритма на
точность аутентификации;
\item проведен ряд экспериментов, которые 
подтвердили практическую применимость системы для защиты интернет-ресурса;
\end{enumerate}

В качестве дальнейших путей развития системы можно рассматривать следующие:

\begin{enumerate}
\item использование алгоритмов компенсации внешнего шума;
\item использование комбинированных подходов к моделированию источника речи
(нейросети, системы на основе нечетких множеств);
\item создание специализированных клиентских приложений для предоставления
возможности использования системы
голосовой аутентификации пользователям мобильных устройств (коммуникаторы,
карманные ПК).
\end{enumerate}

