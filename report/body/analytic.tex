\chapter{Аналитический раздел}

В данном разделе будет дано обоснование поставленной задачи, а также обзор современных подходов к решению проблемы голосовой верификации.

\section{Обоснование задачи}

\section{Обзор существующих решений}

Работу системы верификации говорящего можно разделить на четыре логические стадии: анализ, выделение характерных признаков, моделирование и тест (принятие решения). В данном разделе дан обзор основных методов, используемых при построении современных систем такого рода, для каждой из стадий. Раздел заканчивается краткими выводами и предположениями о направлениях исследований в данной сфере.

\subsection{Анализ речи}

Среди основных известные проблемы в данной области можно выделить следующие (по \cite{Jayanna09overview}):
\begin{itemize}
\item Современные системы распознавания говорящего теряют в производительности, если входной сигнал зашумлен из-за окружения, канала передачи и прочих внешний причин. Важным моментом является попытка улучшить сигнал на этой предварительной стадии (\cite{});
\item Дискриминативная сила моделей теряется при появлении в речи девиантных составляющих, привносимых различными психологическими состояниями говорящего (эмоции, напряжение). Существуют решения, позволяющие компенсировать данные составляющие (\cite{});
\item Для биометрической аутентификации человека современным системам требуется значительное количество речевых данных (порядка десятков минут на человека). В таких условиях возрастает и требовательность системы к вычислительным ресурсам. Применимость технологий верификации по голосу увеличится, если удастся разработать технологии, позволяющие получить удовлетворительную производительность с использованием относительно малого количества данных (10-15 секунд). Существующим решением является использование модели гауссовой смеси с использованием так называемой универсальной фоновой модели (UBM -- \important{Universal Background Model}) \cite{Reynolds00speakerverification}.
\end{itemize}

Рассмотрим более подробно методы и решения, применяемые в существующих системах.

\subsection{Сегментальный анализ}

In this case, speech is analyzed using the frame size and
shift in the range of 10-30 ms to extract the speaker infor-
mation mainly due to the vocal tract. The speaker-specific
vocal tract information may be assumed to be stationary
for all practical analyses and processing when viewed
in frames of size and shift in the range of 10-30 ms [13].
Studies made in [14-19] used segmental analysis to extract
the vocal tract information for speaker recognition.

\subsubsection{Суб-сегментальный анализ}

 Speech analyzed using the frame size and shift in the
range of 3-5 ms is known as sub-segmental analysis [20].
This technique is used mainly to analyze and extract
the characteristics of the excitation source. Since the
excitation source information is relatively fast varying
compared to the vocal tract information, small frame
size and shift are required to best capture the speaker-
specific information, which is the reason for the choice of
3-5 ms for frame size and shift. Studies made in [21-26]
demonstrated that speaker-specific excitation source
information captured using the sub-segmental analysis,
indeed, contains speaker information.


\subsubsection{Супра-сегментальный анализ}

\subsubsection{Вывод}
% сравнение

\subsection{Выделение речевых признаков}

\subsection{Моделирование говорящего}

\subsection{Принятие решения}

\section{Выводы}
